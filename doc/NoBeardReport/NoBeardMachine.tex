%!TEX root = NoBeardReport.tex
\chapter{The \leongage{} Machine}
\lstset{language=AsmDef}
\section{Overview}
The virtual machine being target for \leongage{} programs is a stack machine with instructions of variable length and has the following components. The word width of the \leongage{} machine is four bytes.

\subsection{Program Memory}
The program memory is further denoted as \lstinline$prog[MAX_PROG]$. It is byte addressed with a maximum size of \lstinline$MAX_PROG$. Attempts to access addresses outside the range of 0 to \lstinline$MAX_PROG - 1$ result in a \lstinline$ProgramAddressError$.

\subsection{Data Memory}
The {\em data memory} is byte addressed and storage is done in the following way:

\begin{itemize}
\item Characters are one-byte values and are stored byte-wise into the data memory.
\item Integers are four-byte values and are stored in {\em little endian order} i.e., the lowest significant byte is stored first. Negative integer values are stored as {\em Two's Complement}~\cite{wikipedia_twos_2016}.
\item Booleans are four-byte values and are stored as the integer 0 for false and the integer 1 for true.
\end{itemize}
 The data memory is separated into two parts:
	\begin{itemize}
		\item String constants
		\item Stack frames of the currently running functions
	\end{itemize}
	Figure~\ref{fig:datamemory} shows this. Before a program is started the string constants are stored in the constant memory. On top of this the stack frames are maintained as follows: Every time a function is called a frame is added. It holds data for the function arguments, local variables, some auxiliary data and its expression stack (shortly called stack in the sequel). As soon as the function ends, its frame is removed. A more detailed description of stack frames is given in section~\ref{sec:stackframes}.

\begin{figure}
\begin{center}
\begin{tabular}{p{8em}|p{8em}|}
\cline{2-2}
\parbox[t][3em][t]{8em}{\hfill 0} & String Constants \\[3em] \cline{2-2}
& Stack frame 1 \\[2em] \cline{2-2}
& Stack frame 2 \\[2em] \cline{2-2}
& \ldots \\[2em] \cline{2-2}
\parbox[b][4em][b]{8em}{\hfill MAX\_DATA} & free \\ \cline{2-2}
\end{tabular}
\end{center}
\caption{Data Memory of the \leongage{} Machine}\label{fig:datamemory}
\end{figure}

\subsection{Call Stack}
Since most of the data in the data memory is organized as a stack the {\em call stack} as an abstraction to the data memory is defined. It provides functions to add and remove frames from the stack and to maintain the expression stack.
The expression stack is used to store data needed for each statement. It grows and shrinks as needed and is empty at the end of each statement. The stack is addressed word-wise only. The functions \lstinline$push()$ and \lstinline$pop()$ are used to add and remove values to and from the stack, respectively. It has the following components:

\begin{tabular}{p{.2\textwidth}p{.71\textwidth}}		
	\lstinline$top$ & Address of the start of the last used word on the stack. \\
	\lstinline$fp$ & {\em Frame Pointer}: Address of the first byte of the currently running function's stack frame. \\
\end{tabular}

\subsection{Control Unit}
The control unit is responsible for the proper execution of programs. It executes one machine cycle, i.e., it fetches, decodes and executes the current instruction. In order to do this it uses the following components:

\begin{tabular}{p{.2\textwidth}p{.71\textwidth}}		
	\lstinline$pc$ & {\em Program Counter}: Start address of the next instruction in \lstinline$prog$ to be executed.\\
	\lstinline$ms$ & {\em Machine State}: The \leongage{} machine may have three different states:
	\begin{itemize}
		\item \lstinline$run$: The machine runs
		\item \lstinline$stop$: The machine stops. Usually when the end of program is reached.
		\item \lstinline$error$: Error state
	\end{itemize}
\end{tabular}

\section{Stack Frames}\label{sec:stackframes}
\begin{figure}
\begin{center}
\begin{tabular}{p{8em}|p{4em}|p{15em}}
\parbox[b][1em][b]{8em}{\hfill \cellcolor{Gray}\textcolor{White}{Address}} & \cellcolor{Gray}\textcolor{White}{Content} & \cellcolor{Gray}\textcolor{White}{Remark} \\ \cline{2-2} 
\cline{2-2}
\parbox[t][1em][t]{5em}{\hfill 0} & 0 & frame pointer of frame 0 \\ \cline{2-2}
& \ldots \\ \cline{2-2}
\parbox[t][1em][t]{5em}{\hfill 32} & 13 & local int in frame 0 \\ \cline{2-2}
\parbox[t][1em][t]{5em}{\hfill 36} & 0 & static link to frame 0 (start of frame 1)\\ \cline{2-2}
& \ldots \\ \cline{2-2}
\parbox[t][1em][t]{5em}{\hfill 68} & 17 & local int in frame 1\\ \cline{2-2}
\parbox[t][1em][t]{5em}{\hfill 72} & 42 & local int in frame 1\\ \cline{2-2}
\parbox[t][1em][t]{5em}{\hfill 76} & 36 & static link to frame 1 (start of frame 2) \\ \cline{2-2}
& \ldots \\ \cline{2-2}
\parbox[t][1em][t]{5em}{\hfill 108} & `D' \\ \cline{2-2}
\parbox[t][1em][t]{5em}{\hfill 109} & 61 \\ \cline{2-2}
\parbox[b][4em][b]{8em}{\hfill MAX\_DATA} & free \\ \cline{2-2}
\end{tabular}
\end{center}
\caption{Snapshot of a call stack with three frames}\label{fig:threeframes}
\end{figure}

Considering a concrete scenario as given in figure~\ref{fig:threeframes} we have three frames. Frame~0 starts at address 0. The first 32 bytes of each frame are reserved for administrative data like the static link and the dynamic link to the surrounding frame, the return value, etc. Address~32 holds the value of a local variable in frame~0.

\lstset{language=NoBeardAsm}

At address~36 frame~1 starts with the address to its statically surrounding frame, i.e, the function (or unit or block) represented by frame~0 is defining the function (or block) represented by frame~1. Frame~1 defines two local values at addresses 68 and 72. Now it can be easily verified that an assembler instruction \lstinline$la 0 32$ (see section~\ref{sec:instructions}) loads the value of the local (i.e., relative to the closest frame pointer) address 32. In the concrete example this is address $36 + 32 = 68$.

\section{Binary File Format}
\leongage{} binaries have the extension \lstinline$.no$ (from \leongage{} object file) and have a format as shown in figure~\ref{fig:binaryfileformat}. From byte 0 onwards the machine instructions are stored in a continuous stream. After the final \lstinline$halt$ instruction the stream of string constants follows immediately.

\begin{figure}
\begin{center}
\begin{tabular}{p{8em}|p{8em}|}
\cline{2-2}
\parbox[t][3em][t]{8em}{\hfill 0} & Instructions \\[3em] \cline{2-2}
\parbox[b][2em][b]{8em}{\hfill N} & String Constants \\ \cline{2-2}
\end{tabular}
\end{center}
\caption{\leongage{} Binary File Format}\label{fig:binaryfileformat}
\end{figure}


\section{Runtime Structure of a \leongage{} Program}
\lstset{language=AsmDef}
The \leongage{} Machine follows the following fixed execution cycle:
\begin{enumerate}
	\item Fetch instruction
	\item Decode instruction
	\item Execute instruction
\end{enumerate}
The very first instruction is fetched from \lstinline$prog[startPc]$ where \lstinline$startPc$ has to be provided as an argument when starting the program. From this point of time onwards the program is executed until the machine state changes from {\em run}.

\begin{lstlisting}
runProg(startPc) {
	fp = start byte of first free word in dat;
	top = fp + 28;
	pc = startPc;
	ms = run;
	
	while (ms == run) {
		fetch instruction which starts at prog[pc];
		pc = pc + length of instruction;
		execute instruction
	}
}
\end{lstlisting}

\section{Instructions}\label{sec:instructions}
\leongage{} instructions have a variable length. Every instruction has an opcode and either zero, or one, or two operands. When describing the instructions we use the following conventions:

Each instruction is described in one of the following subsections. The title of the subsection is the mnemonic by which the instruction is identified on assembler level. Then a table follows which shows the size of the instruction and which bytes carry which information. For all instructions the first byte is dedicated to the op code, which is the id by which the instruction is identified on machine language level.

The remaining bytes, if any, are dedicated to the operands of the instruction. When describing these we use the following conventions:

\begin{tabular}{llcp{18.8em}}
Name & Range & Size & Description \\
Literal & 0 ... 65535 & 2 Bytes & Unsigned integer number. \\
Displacement & 0 ... 256 & 1 Byte & Static difference in hierarchy between declaration and usage of an object. \\
DataAddress & 0 ... 65535 & 2 Bytes & Data address relative to the start of its stack frame.
\end{tabular}

Each of these subsections ends with a description of its operation: First a description in human language is given which is then followed by a formal definition.

% ----------------------------- NOP ----------------------------
\subsection{nop}
\subsubsection{Instruction}
\onebyteinstruction{0x00}

\subsubsection{Operation}
Empty instruction. Does nothing

\begin{lstlisting}
nop
\end{lstlisting}

% ----------------------------- LIT ----------------------------
\subsection{lit}
\subsubsection{Instruction}
\threebyteinstruction{0x01}{Literal}

\subsubsection{Operation}
Pushes a value on the expression stack.

\begin{lstlisting}
lit Literal
push(Literal);
\end{lstlisting}

% ----------------------------- LA ----------------------------
\subsection{la}
\subsubsection{Instruction}
\fourbyteinstructiona{0x02}{Displacement}{DataAddress}

\subsubsection{Operation}
Loads an address on the stack.

	\begin{lstlisting}
	la Displacement DataAddress
	base = fp;
	for (i= 0; i < Displacement; i++) {
		base = dat[base ... base + 3];
	}
	push(base + DataAddress);
	\end{lstlisting}

% ----------------------------- LV ----------------------------
\subsection{lv}
\subsubsection{Instruction}
\fourbyteinstructiona{0x03}{Displacement}{DataAddress}

\subsubsection{Operation}
Loads a value on the stack.

	\begin{lstlisting}
	lv Displacement DataAddress
	base = fp;
	for (i = 0; i < Displacement; i++) {
		base = dat[base ... base + 3];
	}
	adr = base + DataAddress;
	push(dat[addr ... addr + 3]);
	\end{lstlisting}

% ----------------------------- LC ----------------------------
\subsection{lc}
\subsubsection{Instruction}
\fourbyteinstructiona{0x04}{Displacement}{DataAddress}

\subsubsection{Operation}
Loads a character on the stack.

	\begin{lstlisting}
	lc Displacement DataAddress
	base = fp;
	for (i = 0; i < Displacement; i++) {
		base = dat[base ... base + 3];
	}
	// fill 3 bytes of zeros to get a full word
	lw = 000dat[base + Address];
	push(lw);
	\end{lstlisting}

% ----------------------------- LVI ----------------------------
\subsection{lvi}
\subsubsection{Instruction}
\fourbyteinstructiona{0x03}{Displacement}{DataAddress}

\subsubsection{Operation}
Loads a value indirectly on the stack.

	\begin{lstlisting}
	lvi Displacement DataAddress
	base = fp;
	for (i = 0; i < Displacement; i++) {
		base = dat[base ... base + 3];
	}
	adr = base + DataAddress;
	ra = dat[addr ... addr + 3];
	push(ra);
	\end{lstlisting}

% ----------------------------- LCI ----------------------------
\subsection{lci}
\subsubsection{Instruction}
\fourbyteinstructiona{0x05}{Displacement}{DataAddress}

\subsubsection{Operation}
Loads a character indirectly on the stack.

	\begin{lstlisting}
	lci Displacement DataAddress
	base = fp;
	for (i = 0; i < Displacement; i++) {
		base = dat[base ... base + 3];
	}
	ra = dat[base + Address];
	// fill 3 bytes of zeros to get a full word
	lw = 000dat[ra];
	push(lw);
	\end{lstlisting}

% ----------------------------- STO ----------------------------
\subsection{sto}
\subsubsection{Instruction}
\onebyteinstruction{0x07}

\subsubsection{Operation}
Stores a value on an address.

	\begin{lstlisting}
	sto
	x = pop();
	a = pop();
	dat[a ... a + 3] = x;
	\end{lstlisting}

% ----------------------------- STC ----------------------------
\subsection{stc}
\subsubsection{Instruction}
\onebyteinstruction{0x08}

\subsubsection{Operation}
Stores a character on an address.

	\begin{lstlisting}
	stc
	x = pop();
	a = pop();
	// Only take the rightmost byte
	dat[a] = 000x;
	\end{lstlisting}

% ----------------------------- ASSN ----------------------------
\subsection{assn}
\subsubsection{Instruction}
\onebyteinstruction{0x0A}

\subsubsection{Operation}
Array assignment.

	\begin{lstlisting}
	assn
	n = pop();
	src = pop();
	dest = pop();
	for (i = 0; i < n; i++)
	   dat[dest + i] = dat[src + i];
	\end{lstlisting}

% ----------------------------- NEG ----------------------------
\subsection{neg}
\subsubsection{Instruction}
\onebyteinstruction{0x0B}

\subsubsection{Operation}
Negates the top of the stack.

	\begin{lstlisting}
	neg
	x = pop();
	push(-x);
	\end{lstlisting}

% ----------------------------- ADD ----------------------------
\subsection{add}
\subsubsection{Instruction}
\onebyteinstruction{0x0C}

\subsubsection{Operation}
Adds the top two values of the stack.

	\begin{lstlisting}
	add
	push(pop() + pop());
	\end{lstlisting}

% ----------------------------- SUB ----------------------------
\subsection{sub}
\subsubsection{Instruction}
\onebyteinstruction{0x0D}

\subsubsection{Operation}
Subtracts the top two values of the stack.

	\begin{lstlisting}
	sub
	y = pop();
	x = pop();
	push(x - y);
	\end{lstlisting}

% ----------------------------- MUL ----------------------------
\subsection{mul}
\subsubsection{Instruction}
\onebyteinstruction{0x0E}

\subsubsection{Operation}
Multiplies the top two values of the stack.

	\begin{lstlisting}
	mul
	push(pop() * pop());
	\end{lstlisting}

% ----------------------------- DIV ----------------------------
\subsection{div}
\subsubsection{Instruction}
\onebyteinstruction{0x0F}

\subsubsection{Operation}
Divides the top two values of the stack.

	\begin{lstlisting}
	div
	y = pop();
	x = pop();
	if (y != 0)
		push(x / y);
	else
		throwDivByZero();
	\end{lstlisting}

% ----------------------------- MOD ----------------------------
\subsection{mod}
\subsubsection{Instruction}
\onebyteinstruction{0x10}

\subsubsection{Operation}
Calculates the remainder of the division of the top values of the stack.

	\begin{lstlisting}
	mod
	y = pop();
	x = pop();
	push(x % y);
	\end{lstlisting}

% ----------------------------- NOT ----------------------------
\subsection{not}
\subsubsection{Instruction}
\onebyteinstruction{0x11}

\subsubsection{Operation}
Calculates the remainder of the division of the top values of the stack.

	\begin{lstlisting}
	not
	x = pop();
	if (x == 0)
	   push(1);
	else
	   push(0);
	\end{lstlisting}

% ----------------------------- REL ----------------------------
\subsection{rel}
\subsubsection{Instruction}
\twobyteinstruction{0x12}{RelOp}

\subsubsection{Operation}
Compares two values of the stack and pushes the result back on the stack. The operand \lstinline$RelOp$ can have six different values:
\begin{itemize}
	\item 0 for encoding $<$ (smaller than)
	\item 1 for encoding $<=$ (smaller or equal than)
	\item 2 for encoding == (equals)
	\item 3 for encoding != (not equals)
	\item 4 for encoding $>=$ (greater or equal than)
	\item 5 for encoding $>$ (greater than)
\end{itemize}
	\begin{lstlisting}
	rel RelOp
	y = pop();
	x = pop();
	switch(RelOp) {
		case 0:
			if (x < y) push(1); else push(0);
			break;
		case 1:
			if (x <= y) push(1); else push(0);
			break;
		case 2:
			if (x == y) push(1); else push(0);
			break;
		case 3:
			if (x != y) push(1); else push(0);
			break;
		case 4:
			if (x >= y) push(1); else push(0);
			break;
		case 5:
			if (x > y) push(1); else push(0);
			break;
	}
	\end{lstlisting}

% ----------------------------- FJMP ----------------------------
\subsection{fjmp}
\subsubsection{Instruction}
\threebyteinstruction{0x16}{NewPc}

\subsubsection{Operation}
Sets \lstinline$pc$ to \lstinline$newPc$ if stack top value is false.

	\begin{lstlisting}
	fjmp newPc
	x = pop();
	if (x == 0)
	   pc = NewPc;
	\end{lstlisting}

% ----------------------------- TJMP ----------------------------
\subsection{tjmp}
\subsubsection{Instruction}
\threebyteinstruction{0x17}{NewPc}

\subsubsection{Operation}
Sets \lstinline$pc$ to \lstinline$newPc$ if stack top value is true.

	\begin{lstlisting}
	tjmp newPc
	x = pop();
	if (x == 1)
	   pc = NewPc;
	\end{lstlisting}

% ----------------------------- JMP ----------------------------
\subsection{jmp}
\subsubsection{Instruction}
\threebyteinstruction{0x18}{NewPc}

\subsubsection{Operation}
Unconditional jump: Sets \lstinline$pc$ to \lstinline$newPc$.

	\begin{lstlisting}
	jmp newPc
	pc = NewPc;
	\end{lstlisting}

% ----------------------------- IN ----------------------------
\subsection{in}\label{sec:in}
\subsubsection{Instruction}
\twobyteinstruction{0x19}{Type}

\subsubsection{Operation}
Reads data from the terminal. Depending on \lstinline$Type$ different data types are read:

\begin{itemize}
	\item 0: An \lstinline$int$ is read and stored at the address on top of the stack. After execution the value 1 is pushed if an integer was read successfully, otherwise 0 is pushed.
	\item 1: A \lstinline$char$ is read and stored at the address on top of the stack. After execution the value 1 is pushed 
	\item 2: a \lstinline$string$ with a specific length is read
\end{itemize}

	\begin{lstlisting}
	in Type
	switch(Type) {
		case 0:
			a = pop();
			n = readInt();
			dat[a ... a + 3] = n;
			break;
		case 1:
			a= pop();
			ch = readChar();
			dat[a] = ch;
			break;
		case 2:
			a = pop();
			strLen = pop();
			
			break;
	}
	\end{lstlisting}

% ----------------------------- OUT ----------------------------
\subsection{out}\label{sec:out}
\subsubsection{Instruction}
\twobyteinstruction{0x1A}{Type}

\subsubsection{Operation}
Writes data to the terminal. Depending on \lstinline$Type$ different data types are printed:

\begin{itemize}
	\item 0: An \lstinline$int$ with a specific column width is printed
	\item 1: A \lstinline$char$ with a specific column width is printed
	\item 2: a \lstinline$string$ with a specific column width is printed
	\item 3: a new line is printed
\end{itemize}

	\begin{lstlisting}
	out Type
	switch(Type) {
		case 0:
			width = pop();
			x = pop();
			// + means string concatenation in the next line
			formatString = "%" + width + "d";
			printf(formatString, x);
			break;
		case 1:
			width= pop();
			x = pop();
			printf("%c", x);
			for (i = 0; i < width - 1; i++)
				printf(" ");
			break;
		case 2:
			width = pop();
			strLen = pop();
			strAddr = pop();
			printf("%s", dat[strAddr ... strAddr + strLen - 1]);
			for (i = n; i < width - 1; i++)
				printf(" ");
			break;
		case 3:
			printf("\n");
			break;
	}
	\end{lstlisting}

% ----------------------------- INC ----------------------------
\subsection{inc}
\subsubsection{Instruction}
\threebyteinstruction{0x1D}{Size}

\subsubsection{Operation}
Increases the size of the stack frame by \lstinline$Size$.

	\begin{lstlisting}
	inc Size
	top += Size;
	\end{lstlisting}

% ----------------------------- HALT ----------------------------
\subsection{halt}
\subsubsection{Instruction}
\onebyteinstruction{0x1F}

\subsubsection{Operation}
Halts the machine.

	\begin{lstlisting}
	halt
	ms = stop;
	\end{lstlisting}

\chapter{Symbol List}

\lstset{language=NoBeard,
	numbers=left,
	tabsize=2
}
\begin{lstlisting}
unit M;
	function A(int a);
		int b;
		
		int function B(char c);
			int d;
		do
				# some code
		done B;
		
		char function C;
			int e;
		do
			# some more code
		done C;
	do
		# some code on A
	done A;
do
	# this is the main of unit M
done M;
\end{lstlisting}

After parsing line 1 the symbol list looks as follows:

\begin{tabular}{lllrrr}
name & kind & type & size & addr & level \\
\hline
M & PROCKIND & UNITTYPE & 0 & 0 & 0
\end{tabular}

After parsing line 2 a snapshot on the symbol list looks like

\begin{tabular}{lllrrr}
name & kind & type & size & addr & level \\
\hline
M & PROCKIND & UNITTYPE & 0 & 0 & 0 \\
A & PROCKIND & UNITTYPE & 0 & 0 & 1 \\
a & PARKIND & SIMINT & 4 & 32 & 2
\end{tabular}

After parsing line 3

\begin{tabular}{lllrrr}
name & kind & type & size & addr & level \\
\hline
M & PROCKIND & UNITTYPE & 0 & 0 & 0 \\
A & PROCKIND & PROCTYPE & 4 & 0 & 1 \\
a & PARKIND & SIMINT & 4 & 32 & 2 \\
b & VARKIND & SIMINT & 4 & 36 & 2
\end{tabular}

After parsing line 6 being somewhere between line 7 and the end of line 9.

\begin{tabular}{lllrrr}
name & kind & type & size & addr & level \\
\hline
M & PROCKIND & UNITTYPE & 0 & 0 & 0 \\
A & PROCKIND & PROCTYPE & 4 & 0 & 1 \\
a & PARKIND & SIMINT & 4 & 32 & 2 \\
b & VARKIND & SIMINT & 4 & 36 & 2 \\
B & PROCKIND & PROCTYPE & 0 & 0 & 2 \\
c & PARKIND & SIMCHAR & 1 & 32 & 3 \\
d & VARKIND & SIMINT & 4 & 36 & 3
\end{tabular}
