\chapter{The NoBeard Machine Architecture}
\lstset{language=NoBeardAsm}
\section{The NoBeard Machine}
The NoBeard Machine is a virtual machine with an instruction set of 31 instructions which is pretty easy to understand compared to instruction sets of real life machines which have significant larger instruction sets. The machine is purely stack based such that the structure of each instruction is easy to grasp and to follow. The machine has a word width of four bytes and is being target for NoBeard programs. The NoBeard Machine consists of the following components:
\begin{itemize}
	\item Program Memory
	\item Data Memory
	\item Call Stack
	\item Control Unit
\end{itemize}
Figure~\ref{fig:componentsOfNbM} shows these components and how they are related. The components will be described in the following sections.

\begin{figure}
	\centering
	\includegraphics[scale=.62]{images/componentsOfNbM.png}
	\caption{Components of the NoBeardMachine Architecture}
	\label{fig:componentsOfNbM}
\end{figure}
\subsection{Program Memory}
The program memory stores instructions with their corresponding opcodes and operands. The memory is byte addressed with a specified  maximum size. Memory access outside the valid range lead to a \lstinline$ProgramAddressError$.

\subsection{Data Memory} \label{sec:dataMemory} 
The data memory is a byte addressed storage and stores variables as follows:

\begin{itemize}
\item \textbf{Characters} are one-byte values and consume exactly one byte in memory, i.e., no alignment is done. 
\item \textbf{Integers }are four-byte values and stored in little endian order. Negative integer values are stored as Two’s Complement (see \cite{wikipedia_twos_2018}).
\item \textbf{Booleans }are four-bytes values and are stored as the integer 0 for false and the integer 1 for true.
\end{itemize}

\begin{figure}[H]
\begin{center}
\begin{tabular}{p{8em}|p{8em}|}
\cline{2-2}
\parbox[t][3em][t]{8em}{\hfill 0} & String Constants \\[3em] \cline{2-2}
& Stack frame 1 \\[2em] \cline{2-2}
& Stack frame 2 \\[2em] \cline{2-2}
& \ldots \\[2em] \cline{2-2}
\parbox[b][4em][b]{8em}{\hfill MAX\_DATA} & free \\ \cline{2-2}
\end{tabular}
\end{center}
\caption{Data Memory of the NoBeard Machine}\label{fig:datamemory}
\end{figure}

As figure~\ref{fig:datamemory} shows, the data memory is separated into two parts, string constants and to stack frames of the currently running functions. 

\subsubsection{String Constants}
String constants are in the top segment of the data memory. It includes all strings that have to be printed on the console. The storing of string constants happens exactly after the opening of a binary file on the virtual machine. 

\subsubsection{Stack Frames}
After the constant memory the stack frames are maintained. Whenever a function gets called a new frame is added. It holds data for the function arguments, local variables and its expression stack. As soon as a function ends, its frame is removed. 

\begin{figure}[H]
\begin{center}
\begin{tabular}{p{8em}|p{4em}|p{15em}}
\parbox[b][1em][b]{8em}{\hfill \cellcolor{Gray}\textcolor{White}{Address}} & \cellcolor{Gray}\textcolor{White}{Content} & \cellcolor{Gray}\textcolor{White}{Remark} \\ \cline{2-2} 
\cline{2-2}
\parbox[t][1em][t]{5em}{\hfill 0} & 0 & frame pointer of frame 0 \\ \cline{2-2}
& \ldots \\ \cline{2-2}
\parbox[t][1em][t]{5em}{\hfill 32} & 13 & local int in frame 0 \\ \cline{2-2}
\parbox[t][1em][t]{5em}{\hfill 36} & 0 & static link to frame 0 (start of frame 1)\\ \cline{2-2}
& \ldots \\ \cline{2-2}
\parbox[t][1em][t]{5em}{\hfill 68} & 17 & local int in frame 1\\ \cline{2-2}
\parbox[t][1em][t]{5em}{\hfill 72} & 42 & local int in frame 1\\ \cline{2-2}
\parbox[t][1em][t]{5em}{\hfill 76} & 36 & static link to frame 1 (start of frame 2) \\ \cline{2-2}
& \ldots \\ \cline{2-2}
\parbox[t][1em][t]{5em}{\hfill 108} & `D' \\ \cline{2-2}
\parbox[t][1em][t]{5em}{\hfill 109} & 61 \\ \cline{2-2}
\parbox[b][4em][b]{8em}{\hfill MAX\_DATA} & free \\ \cline{2-2}
\end{tabular}
\end{center}
\caption{Snapshot of a call stack with three frames}\label{fig:threeframes}
\end{figure}

Figure~\ref{fig:threeframes} shows a pretty good example from \cite{bauer_p._2017}. Here we can see that the memory is working with three frames. Frame~0 starts at address 0. The first 32 bytes of each frame are reserved for administrative data like the static link and the dynamic link to the surrounding frame, the return value, etc. Address~32 holds the value of a local variable in frame~0.
At address~36 frame~1 starts with the address to its statically surrounding frame, i.e, the function (or unit or block) represented by frame~0 is defining the function (or block) represented by frame~1. Frame~1 defines two local values at addresses 68 and 72. 

\subsection{Call Stack}
By structuring the data memory as a stack the call stack is needed as an abstraction to the data memory. With the help of different functions the call stack is able to add and remove frames from the stack and to maintain the expression stack. Data needed for each statement gets stored in the expression stack. It grows and shrinks as needed and is empty at the end of each NoBeard statement. The stack is addressed word-wise only. Functions like \lstinline$push()$, \lstinline$peek()$ and \lstinline$pop()$ are provided for the maintenance of values on the stack. The call stack has the two major components:

\begin{itemize}
\item \textbf{Stack Pointer: }Address of the start of the last used word on the stack. 
\item \textbf{Frame Pointer: }Address of the first byte of the currently running function's stack frame. 
\end{itemize}

\subsection{Control Unit}
The control unit is responsible for the program work flow. It executes one machine cycle in three steps, it fetches, decodes and operates the current instruction. Depending on some instruction, the control unit also affect the state of the machine. To achieve these steps it has to work with the following components:

\begin{itemize}
\item \textbf{Program Counter: }Start address of the next instruction to be executed.
\item \textbf{Machine State: }The NoBeard machine has four different states and is always in one of them. 
	\begin{itemize}
		\item \lstinline$running$: The machine runs
		\item \lstinline$stopped$: The machine stops. Usually when the end of program is reached.
		\item \lstinline$blocked$: The machine pauses. Mostly when a breakpoint is placed by the user.
		\item \lstinline$error$: Error state
	\end{itemize}
\end{itemize}
As already mentioned the machine has a firmly defined execution cycle:

\begin{enumerate}
\item Fetch instruction
\item Decode instruction
\item Execute instruction
\end{enumerate}

The very first instruction is fetched from a specified starting program counter which is provided as an argument when starting the program. From this point of time onwards the program is running until the machine state changes from run. There are two options to interrupt the machine from running state. First, if it gets interrupted by a breakpoint typically set by the debugger and second, if a \lstinline$halt$ instruction gets executed.

\section{Binary File Format}
The virtual machine runs only NoBeard object files with extension \lstinline$.no$ which can be generated by the NoBeard Assembler or NoBeard Compiler. As figure~\ref{fig:binaryfileformat} shows NoBeard binaries are separated into three parts, a header part, a string storage and a program segment.
The first six bytes are reserved for the header part which holds information about the file. This information includes the file identifier and the version of the file. After the header follows the string storage where a stream of constants are stored. Finally, the program segment which is organized like the string storage deals with the storing of machine instructions. 
\begin{figure}[h]
\begin{center}
\begin{tabular}{p{1em}|p{10em}|}
\cline{2-2}
\parbox[c][3em][t]{1em}{\hfill 0} & Header \\ \cline{2-2}
\parbox[c][4em][t]{1em}{\hfill 6} & String storage \\ \cline{2-2}
\parbox[c][5em][b]{1em}{\hfill N} & Program segment \\ \cline{2-2}
\end{tabular}
\end{center}
\caption{NoBeard Binary File Format}\label{fig:binaryfileformat}
\end{figure}

\section{Instructions}\label{sec:instructions}
NoBeard instructions are of a different length and each has an opcode and operands of an amount between 0 and 2. The first byte of all instructions is reserved for the opcode, which is the identifier used to identify the instruction on machine language level. The remaining bytes, if any, are assigned to the operand(s) of the instruction.
Each of the following subsections explains these instructions in four categories. The underlined title is the shorthand that identifies the instruction on assembler level. Then follows a table showing the size of the instruction and which bytes carry which information. Finally, each one has also a short explanation in human language.

\subsection{Load and Store Instructions}
% ----------------------------- LIT ----------------------------
\subsubsection{lit}
\threebyteinstruction{0x01}{Literal}

\paragraph{Operation:}
Pushes a value on the expression stack.

% ----------------------------- LA ----------------------------
\subsubsection{la}
\fourbyteinstructiona{0x02}{Displacement}{DataAddress}

\paragraph{Operation:}
Loads an address on the stack.

% ----------------------------- LV ----------------------------
\subsubsection{lv}
\fourbyteinstructiona{0x03}{Displacement}{DataAddress}

\paragraph{Operation:}
Loads a value on the stack.

% ----------------------------- LC ----------------------------
\subsubsection{lc}
\fourbyteinstructiona{0x04}{Displacement}{DataAddress}

\paragraph{Operation:}
Loads a character on the stack.

% ----------------------------- LVI ----------------------------
\subsubsection{lvi}
\fourbyteinstructiona{0x05}{Displacement}{DataAddress}

\paragraph{Operation:}
Loads a value indirectly on the stack.

% ----------------------------- LCI ----------------------------
\subsubsection{lci}
\fourbyteinstructiona{0x06}{Displacement}{DataAddress}

\paragraph{Operation:}
Loads a character indirectly on the stack.

% ----------------------------- INC ----------------------------
\subsubsection{inc}
\threebyteinstruction{0x1D}{Size}

\paragraph{Operation:}
Increases the size of the stack frame by \lstinline$Size$.

% ----------------------------- STO ----------------------------
\subsubsection{sto}
\onebyteinstruction{0x07}

\paragraph{Operation:}
Stores a value on an address.

% ----------------------------- STC ----------------------------
\subsubsection{stc}
\onebyteinstruction{0x08}

\paragraph{Operation:}
Stores a character on an address.

\subsection{Integer Instructions}


% ----------------------------- NEG ----------------------------
\subsubsection{neg}
\onebyteinstruction{0x0B}

\paragraph{Operation:}
Negates the top of the stack.

% ----------------------------- ADD ----------------------------
\subsubsection{add}
\onebyteinstruction{0x0C}

\paragraph{Operation:}
Adds the top two values of the stack.

% ----------------------------- SUB ----------------------------
\subsubsection{sub}
\onebyteinstruction{0x0D}

\paragraph{Operation:}
Subtracts the top two values of the stack.

% ----------------------------- MUL ----------------------------
\subsubsection{mul}
\onebyteinstruction{0x0E}

\paragraph{Operation:}
Multiplies the top two values of the stack.

% ----------------------------- DIV ----------------------------
\subsubsection{div}
\onebyteinstruction{0x0F}

\paragraph{Operation:}
Divides the top two values of the stack.

% ----------------------------- MOD ----------------------------
\subsubsection{mod}
\onebyteinstruction{0x10}

\paragraph{Operation:}
Calculates the remainder of the division of the top values of the stack.

% ----------------------------- NOT ----------------------------
\subsubsection{not}
\onebyteinstruction{0x11}

\paragraph{Operation:}
Calculates the remainder of the division of the top values of the stack.

\subsection{Control Flow Instructions}

% ----------------------------- FJMP ----------------------------
\subsubsection{fjmp}
\threebyteinstruction{0x16}{NewPc}

\paragraph{Operation:}
Sets \lstinline$pc$ to \lstinline$newPc$ if stack top value is false.

% ----------------------------- TJMP ----------------------------
\subsubsection{tjmp}
\threebyteinstruction{0x17}{NewPc}

\paragraph{Operation:}
Sets \lstinline$pc$ to \lstinline$newPc$ if stack top value is true.

% ----------------------------- JMP ----------------------------
\subsubsection{jmp}
\threebyteinstruction{0x18}{NewPc}

\paragraph{Operation:}
Unconditional jump: Sets \lstinline$pc$ to \lstinline$newPc$.

% ----------------------------- BREAK----------------------------
\subsubsection{break}
\onebyteinstruction{0x20}

\paragraph{Operation:}
Breaks the machine. Especially used for the debugging function.

% ----------------------------- HALT ----------------------------
\subsubsection{halt}
\onebyteinstruction{0x1F}

\paragraph{Operation:}
Halts the machine.

\subsection{IO-Instructions}

% ----------------------------- IN ----------------------------
\subsubsection{in}\label{sec:in}
\twobyteinstruction{0x19}{Type}

\paragraph{Operation:}
Reads data from the terminal. Depending on \lstinline$Type$ different data types are read:

\begin{itemize}
	\item 0: An \lstinline$int$ is read and stored at the address on top of the stack. After execution the value 1 is pushed if an integer was read successfully, otherwise 0 is pushed.
	\item 1: A \lstinline$char$ is read and stored at the address on top of the stack. After execution the value 1 is pushed 
	\item 2: a \lstinline$string$ with a specific length is read
\end{itemize}

% ----------------------------- OUT ----------------------------
\subsubsection{out}\label{sec:out}
\twobyteinstruction{0x1A}{Type}

\paragraph{Operation:}
Writes data to the terminal. Depending on \lstinline$Type$ different data types are printed:

\begin{itemize}
	\item 0: An \lstinline$int$ with a specific column width is printed
	\item 1: A \lstinline$char$ with a specific column width is printed
	\item 2: a \lstinline$string$ with a specific column width is printed
	\item 3: a new line is printed
\end{itemize}
	
\section{NoBeard Assembler}
To write programs for the NoBeard machine an Assembler is provided. NoBeard Assembler files are separated in two blocks, which is called the string constants and the assembler program. The files have the extensions \lstinline$.na$ for NoBeard Assembler. The string constants are stored between two double quotes and has to be located at the beginning of the file. There is no way to address a single constant. So, if we use a string constant in the assembler program, we have to specify the starting address of the string constant and the length needed in the program. Assembler programs  contain a sequence of assembler instructions like \lstinline$lit$ for load or \lstinline$out$ for print. After the opcode of the instruction follows the operands, if they are needed as already described in section \ref{sec:instructions}.