\chapter{The NoBeard Machine Architecture}
\section{The NoBeard Machine}
It is a virtual machine with an instruction set of 31 instructions which is pretty easy to grasp compared to the instruction set sizes of real life machines. The machine is purely stack based such that the structure of each instruction is easy to understand and to follow. The word width of the machine is four bytes. The NoBeard Machine consists of the following components.
\subsection{Program Memory}
It is byte address with a specified maximum size. Addresses access outside the range of 0 to the maximum size -1 result in a “ProgramAddressError”.
\subsection{Data Memory} 
\label{ssec:dataMemory}
The Data memory is a storage which is byte addressed and stores variables in the following way:
\begin{itemize}
\item \textbf{Characters }are one-byte values are stored byte wise into the data memory. 
\item \textbf{Integers }are four-byte values and are stored in little endian order. Negative integer values are stored as Two’s Complement.
\item \textbf{Booleans }are four-bytes values and are stored as the integer 0 for false and the integer 1 for true.
\end{itemize}
The data memory is separated into two parts, string constants and to stack frames of the currently running functions. Before starting a program on the virtual machine string constants are stored in the constant memory. For every execution of a function a new frame is added. It holds data for the functions arguments, local variables and its expression stack. As soon as functions ends, its frame is removed. 
\subsection{Call Stack}
It provides functions to add and remove frames from the stack and to maintain the expression stack. Data of each statement is stored in the expression stack. It grows and shrinks as needed and is empty at the end of each statement. The stack is addressed word-wise only. 
\subsection{Control Unit}
It is responsible for the program work flow. It executes one machine cycle in three steps, it fetches, decodes and operates the current instruction. Depending on some instruction, the control unit also affect the state of the machine. 
\subsection{Binary File Format}
The virtual machine runs only NoBeard object files with extension .no which can be generated by an NoBeard Assembler or NoBeard Compiler. From the first byte onwards the machine instructions are stored in a continous stream. After a final “halt” instruction the stream of string constants is stored. 
\subsection{Runtime Structure of NoBeard Program}
The machine has a firmly defined execution cycle:
\begin{enumerate}
\item Fetch instruction
\item Decode instruction
\item Execute instruction
\end{enumerate}
The very first instruction is fetched from a specified starting program counter which is provided as an argument when starting the program. From this point of time forwards the program is running until the machine state changes from run. 
\subsection{Instructions}
NoBeard instructions have a variable length and each one has an opcode and operands with an amount of between 0 and 2. For all instructions the first byte is dedicated to the op code, which is the id by which the instruction is identified on machine language level. The remaining bytes, if any, are dedicated to the operands of the instruction. 
\section{NoBeard Assembler}
To write programs for the NoBeard machine an Assembler is provided. NoBeard Assembler files are separated in two blocks, which is called the string constants and the assembler program. The files have the extensions .na for NoBeard Assembler. The string constants are stored between two double quotes and has to be located at the beginning of the file. There is no possibility to address one single constant, so when using a string constant in the assembler program one has to provide the start address of the string constant and the length needed in the program.  Assembler programs contains a sequence of assembler instructions. The programmer has to follow the instruction format as shown in the figure