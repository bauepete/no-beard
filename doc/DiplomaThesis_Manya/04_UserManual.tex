\chapter{User Manual}
This chapter describes how to use the graphical user interface of the NoBeard virtual machine and serves at the same time as a user manual.
With the NoBeardMachine users are able to load and run NoBeard object files with just a view clicks. The integrated debugger of the virtual machine gives them also the possibility to debug these programs by setting breakpoints. Another big feature is the data visualization which gives users a whole view of the data memory.
\section{The UI Components}
As shown in figure~\ref{fig:components} the NoBeardMachine contains the following five windows:
\begin{figure}[h] 
	\centering
	\includegraphics[scale=.87]{images/screenshot-0.png}
	\caption{Components of the NoBeardMachine}
	\label{fig:components}
\end{figure}
\subsection{Control Window}
This window consists of five buttons that gives the possibility to open, run, step, stop or jump in programs. 
\begin{itemize}
\item \textbf{Open file: }Opens up a file chooser dialog where the target object file can be selected.
\item \textbf{Run: }Executes the program from the first byte.
\item \textbf{Step: }Gives the opportunity to step one single line further in the program flow. This button is only in a "blocked" state available.
\item \textbf{Continue: }When the program execution gets interrupted by a breakpoint this button enables users to jump to the next breakpoint. If there is no breakpoints left it continues the process until the end.
\item \textbf{Stop: }Stops the execution and sets the machine into the "stopped" state.
\end{itemize}
\subsection{Output Window}
It is a non-editable TextArea which simulates a terminal. Program outputs that are coming from an "OUT" instruction are shown here. Submitted inputs are also visible here after a successfully submit.
\subsection{Program Window}
This window shows assembler instructions with the belonging addresses in a VBox. Every line also gets a checkbox that is responsible for setting and removing breakpoints.
\subsection{Data Memory Window}
A ListView filled with raw data from the data memory. Every line of the ListView contains four byte data with the belonging addresses. Each byte can be converted to a character. Even a whole line can be translated to four characters or to a single four byte integer. Data with blue background color highlights the frame pointer. Red background color stands for the stack pointer.
\subsection{Input Window}
The Input window is a TextField where inputs from users can be handled. It is only enabled when the machine is executing an "IN" instruction. To submit an input, the "ENTER" key has to be pressed and then the entered text will be attached to the Output window.
\section{Loading and Running a Program}
After starting the NoBeardMachine, a NoBeard object file has to be loaded by a click on the “Open File” button. Than a file chooser dialog should appear where the user can choose the desired file. Afterward, the window shows the assembler code and the title of the opened program which is now executable.

The Program window lists the assembler instructions with the belonging addresses and operators. By hitting the “Run” button, the machine executes the loaded program. Output results can be seen in the Output window. If the program runs against an input instruction, the machine stops and requests the user for an input which can be done at the Input window. To submit an input, the Enter key has to be pressed.
\begin{figure}[h] 
	\centering
	\includegraphics[scale=.85]{images/screenshot-1.png}
	\caption{Executed "Hello World" program}
\end{figure}

\section{Debugging}
To debug a program the user has to set breakpoints which interrupts the program flow and allows the user to inspect the current state of the running program. These breakpoints can be placed by clicking on the address with the instruction where the program flow has to be interrupted. 

After an interruption of a breakpoint, the user is able to step one line further or jump to the next breakpoint. Stepping is handled by the “Step” button as shown in figure~\ref{fig:debugging}. By clicking the “Continue” button, the program runs from the current line until the next breakpoint. If there is not any breakpoint left from the current line, it runs until the end of the program. Optionally, the user is able to stop the program during the execution. This could be achieved by clicking the “Stop” button.
\begin{figure}[h] 
	\centering
	\includegraphics[scale=.85]{images/screenshot-2.png}
	\caption{Debugging the "Hello World" program}
	\label{fig:debugging}
\end{figure}

\section{Data Visualization}
On the right side of the window is the visualisation of the data memory in a ListView form. This window lists data byte wise from the data memory of the machine. Each line of the ListView has a content of an address given in decimal notation following by four bytes raw data. The memory is separated in two parts. The list starts on the top with the string constants followed by stack frames of the currently running functions. While the frame pointer is highlighted with a blue background, the stack pointer is signed with a red background. The user has also the possibility to convert raw data to characters or integers.
With a right click on the selected line of the list, a context menu could be opened. This menu includes following functions:
\begin{itemize}
\item showing integer,
\item characters of the whole line, 
\item a single character 
\item or converting back the line to raw data. 
\end{itemize}
\begin{figure}[h] 
	\centering
	\includegraphics[scale=.60]{images/screenshot-3.png}
	\caption{Converting data to a character}
	\label{fig:convertToChar}
\end{figure}
\subsection{Converting Data to Character}
To view a character of a one-byte data as shown in figure~\ref{fig:convertToChar}, the following steps has to be done:
\begin{enumerate}
\item Select the line where the one-byte value is located 
\item Right click on the value to be converted
\item Click on "View Char" 
\end{enumerate}
Now the value at the given address is translated to an alphanumeric character. 
\subsection{Converting Data to Integer}
The translation of an integer takes four bytes that means, it takes all values of a whole row and translates them to a single integer.  
\begin{figure}[h] 
	\centering
	\includegraphics[scale=.85]{images/screenshot-4.png}
	\caption{Converting data to a single integer}
	\label{fig:convertToInt}
\end{figure}
\begin{enumerate}
\item Select the row with the four-byte value 
\item Right click on the line
\item Click on "View Integer" 
\end{enumerate}
Figure~\ref{fig:convertToInt} shows an example of a translated integer which is on the same place as the stack pointer, precisely at the address 64. A detailed description of the implementation can be found in chapter~\ref{sec:implementationOfDataVisualisation}.
\subsection{Multiple Conversion}
To make the conversion from raw data to alphanumeric characters or integers more easy and fast, a multi selection function is available for the list of the data memory.
To convert multiple data rows, the user has to hold the "Ctrl" key and select the specified rows with the left mouse click. Chosen lines will get highlighted with a light grey background color. As figure~\ref{fig:multipleConversion} shows, the selected rows with grey background are successfully concerted to some characters.
\begin{figure}[h] 
	\centering
	\includegraphics[scale=.85]{images/screenshot-5.png}
	\caption{Multiple conversion of data}
	\label{fig:multipleConversion}
\end{figure}
