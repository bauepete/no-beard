\section*{Declaration of Academic Honesty}
Hereby, I declare that I have composed the presented paper independently on my own and without any other resources than the ones indicated. All thoughts taken directly or indirectly from external sources are properly denoted as such.

This paper has neither been previously submitted to another authority nor has it been published yet. \\[1em]
Leonding, \duedateen \\[5em]
\ifthenelse{\isundefined{\firstauthor}}{}{\firstauthor}
\ifthenelse{\isundefined{\secondauthor}}{}{\kern-1ex, \secondauthor}
\ifthenelse{\isundefined{\thirdauthor}}{}{\kern-1ex, \thirdauthor} \\[5em]

\begin{otherlanguage}{german}
\section*{Eidesstattliche Erklärung}
Hiermit erkläre ich an Eides statt, dass ich die vorgelegte Diplomarbeit selbstständig und ohne Benutzung anderer als der angegebenen Hilfsmittel angefertigt habe. Gedanken, die aus fremden Quellen direkt oder indirekt übernommen wurden, sind als solche gekennzeichnet.

Die Arbeit wurde bisher in gleicher oder ähnlicher Weise keiner anderen Prüfungsbehörde vorgelegt und auch noch nicht veröffentlicht. \\[1em]
Leonding, am \duedatede \\[5em]
\ifthenelse{\isundefined{\firstauthor}}{}{\firstauthor}
\ifthenelse{\isundefined{\secondauthor}}{}{\kern-1ex, \secondauthor}
\ifthenelse{\isundefined{\thirdauthor}}{}{\kern-1ex, \thirdauthor} \\[5em]
\end{otherlanguage}

\begin{abstract}
The target of this diploma thesis is to extend an available system programming environment which is called the NoBeard project. The project is developed to enable students to gain some experiences in the field of system programming by coding on assembler level with basic instructions. In addition, the reader of this thesis gets a basic knowledge about formal languages and their uses.

Initially, it contained three big components:
\begin{itemize}
\item {\em The NoBeard Machine:} A stack based virtual machine with a small set of instructions. The main task of the machine is to execute NoBeard object files.
\item {\em The NoBeard Assembler:} Provided to write programs for the NoBeard Machine.
\item {\em The NoBeard Compiler:} Includes a dedicated programming language and the corresponding compiler to simplify the writing of codes for the NoBeard Machine.
\end{itemize}

The purpose of this thesis was to develop a proper concept of a graphical user interface for the virtual machine.
The concept has to focus the didactic aims to enable users to explore the execution cycle of an assembler instruction, the execution of programs on assembler level, the monitoring of stack frames, the expression stack etc\ldots 

Furthermore, the system is extended with some debugger functions such as setting break points, stepping on assembler instruction level etc\ldots

The initial version of the NoBeard project was developed in Java, therefore the graphical user interface had to be implemented with the Java FX framework.
Primary, it was conceived with the NetBeans IDE on the ant build system, however the entire project was migrated to Maven and further developed using the IntelliJ IDEA environment.
\end{abstract}

\begin{otherlanguage}{german}
\begin{abstract}
An dieser Stelle wird beschrieben, worum es in der Diplomarbeit geht. Die Zusammenfassung soll kurz und prägnant sein und den Umfang einer Seite nicht übersteigen. Weiters ist zu beachten, dass hier keine Kapitel oder Abschnitte zur Strukturierung verwendet werden. Die Verwendung von Absätzen ist zulässig. Wenn notwendig, können auch Aufzählungslisten verwendet werden. Dabei ist aber zu beachten, dass auch in der Zusammenfassung vollständige Sätze gefordert sind.

Bezüglich des Inhalts sollen folgende Punkte in der Zusammenfassung vorkommen: 

\begin{itemize}
\item {\em Aufgabenstellung:} Von welchem Wissenstand kann man im Umfeld der Aufgabenstellung ausgehen? Was ist das Ziel des Projekts? Wer kann die Ergebnisse der Arbeit benutzen?
\item {\em Umsetzung:} Welche fachtheoretischen oder -praktischen Methoden wurden bei der Umsetzung verwendet?
\item {\em Ergebnisse:} Was ist das endgültige Ergebnis der Arbeit?
\end{itemize}
Diese Liste soll als Sammlung von inhaltlichen Punkten für die Zusammenfassung verstanden werden. Die konkrete Gliederung und Reihung der Punkte ist den Autoren überlassen. Zu beachten ist, dass der/die LeserIn beim Lesen dieses Teils Lust bekommt, diese Arbeit weiter zu lesen.

Abschließend soll die Zusammenfassung noch ein Foto zeigen, das das beschriebene Projekt am besten repräsentiert. Das folgende Bild zeigt Leslie Lamport, den Erfinder von \LaTeX.

\begin{flushright}
	\includegraphics[scale=.25]{images/leslie_lamport.jpg}
\end{flushright}

\end{abstract}
\end{otherlanguage}

\section*{Acknowledgments}
If you feel like saying thanks to your grandma and/or other relatives.
